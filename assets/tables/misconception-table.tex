\bgroup
\def\arraystretch{1.5}
\begin{table}[htbp]
    \centering
    \caption{Myths and Realities about Reentrancy Attacks, plus the Implications for Detection Techniques}
    \begin{footnotesize}
    \begin{tabular}{|p{0.7cm}|p{2.2cm}|p{2.3cm}|p{5.8cm}|}
        \hline
        \textbf{Sec} & \textbf{Myth} & \textbf{Reality} & \textbf{Explanation} \\
        \hline
        \ref{sec:results-entry} \ref{sec:results-financial} \ref{sec:results-timeline} & They decrease because of reentrancy guard adoption~\cite{verheijke2022exploratory}. & They remain prevalent to this day. & After the DAO attack, reentrancy attacks ceased for over two years. Following the SpankChain attack (Oct 19, 2018)~\cite{attack-report:spankchain}, another 1.5 years passed without incidents. However, attacks surged in mid-2021 and have persisted at 1.5 incidents per month on average ever since (see Figures~\ref{fig:entry-point},~\ref{fig:fund-loss},~\ref{fig:attack-timeline}). \\
        \hline
        \ref{sec:results-entry} & They only target the \lstinline|fallback| function \cite{smart-contract-security-field-guide}. & \nohyphen{They utilize various entry points~\cite{immunefi-ultimate-guide}.} & While the DAO attack used the \lstinline|fallback| function, modern attacks target malicious tokens, ERC hooks, and application-specific hooks. The \lstinline|fallback| function accounts for only 30.4\% of analyzed attacks. \\
        \hline
        \ref{sec:results-entry} & They only exploit native tokens and fungible tokens \cite{rahimian2021tokenhook, nelaturu2022correct}. & They also target other types of tokens, including NFTs and hybrid tokens. & Although most attacks involve fungible tokens, NFTs such as CreatureToadz~\cite{attack-report:creature-toadz} and HypeBeans~\cite{attack-report:hype-beans}, along with hybrid tokens~\cite{attack-report:revest-finance}, have also been exploited. \\
        \hline
        \ref{sec:results-strategy} & They all involve recursive withdrawal~\cite{smart-contract-security-field-guide}. & Reentrant (recursive) withdrawal is only one of many strategies attackers use. & The first two reentrancy attacks before 2020 did use reentrant withdrawal. However, the third attack (imBTC on April 18, 2020~\cite{attack-report:imbtc}) applied a different one: price manipulation. Our analysis identifies seven distinct attack strategies. \\ 
        \hline
        \ref{sec:results-callbacks} & Reentrancy is inherently malicious \cite{reddit-disable-reentrant-behavior}. & Reentrancy can also be legitimate. & Reentrancy is a common pattern that could be used to support modular and flexible designs. This is increasingly prevalent with the rise of cross-contract interactions, especially after the adoption of proxy contracts for upgradability. \\
        \hline
    \end{tabular}
    \end{footnotesize}
    \vspace{-3mm}
    \label{tab:2}
\end{table}
\egroup

