\bgroup
\def\arraystretch{1.4}
\begin{table}[htbp]
    \centering
    \caption{Case Study Summary for Qualitative Analysis}
    \vspace{-1mm}
    \begin{footnotesize}
        \newcommand{\tableheader}[1]{\textbf{\changefont[9pt]{#1}}}
        \begin{tabular}{|p{0.25cm}|p{2cm}|p{1.6cm}|p{7cm}|}
            \hline
            \tableheader{No.}          & \tableheader{Name}              & \tableheader{Date} & \tableheader{Vulnerability Description}                                                                                                                                                                                                               \\
            \hline
            \hyperref[case-study:1]{1} & ChainPaint Attack               & 2024-02-12         & Old-school pattern using user balance updates after \lstinline$call.value$ transfers, similar to the vulnerability exploited in The DAO attack.                                                                                                       \\
            \hline
            \hyperref[case-study:2]{2} & Peapods Finance Whitehat Attack & 2023-12-13         & \multirow{3}{7.6cm}{Flashloan functions lacked reentrancy guards and checked repayment by contract balance only, enabling attackers to register deposits made in reentrancy as repayment.}                                                            \\
            \cline{2-3}
                                       & 0x0 Attack                      & 2023-10-27         &                                                                                                                                                                                                                                                       \\
            \cline{2-3}
                                       & DFX Finance Attack              & 2022-11-10         &                                                                                                                                                                                                                                                       \\
            \hline
            \hyperref[case-study:3]{3} & Predy Finance Attack            & 2024-05-14         & Reentrancy used as a feature, allowing callers to reenter \lstinline$PredyPool$ via a custom callback function and move funds with the \lstinline$take$ function. But the failure to check each individual pair's balance led to a successful attack. \\
            \hline
            \hyperref[case-study:4]{4} & Revest Finance Attack           & 2022-03-27         & Stale \lstinline$FNFTHandler.fnftsCreated$ variable enabled reentrancy during minting, allowing attackers to overwrite data of newly minted FNFTs.                                                                                                    \\
            \hline
            \hyperref[case-study:5]{5} & Omni Protocol Attack            & 2022-07-10         & Bypassed collateral checks in \lstinline$withdrawERC721$ by manipulating configurations through a reentrant call, allowing withdrawal of NFTs still used as collateral.                                                                               \\
            \hline
            \hyperref[case-study:6]{6} & BNB Brokers Rug Pull            & 2022-04-27         & \multirow{2}{7.6cm}{Suspected rug pull: used reentrancy backdoor to drain the contract's funds, followed by deactivation of social media and shutdown of project websites.}                                                                           \\
            \cline{2-3}
                                       & Barley Finance Rug Pull         & 2024-01-28         &                                                                                                                                                                                                                                                       \\
            \hline
            \hyperref[case-study:7]{7} & GoodDollar Attack               & 2023-12-26         & Public accessibility of a privileged function allowed the attacker to exploit reentrancy to simulate interest collection, causing artificial appreciation of G\$X tokens.                                                                             \\
            \hline
            \hyperref[case-study:8]{8} & Curve Attack                    & 2023-07-30         & Bug in the Vyper compiler invalidated its built-in reentrancy guard, rendering Curve vulnerable to price manipulation reentrancy attack.                                                                                                              \\
            \hline
            \hyperref[case-study:9]{9} & Terra Attack                    & 2024-07-31         & Mishandled timeout message in Terra's IBC implementation enabled attackers to mint additional legitimate tokens at no cost.                                                                                                                           \\
            \hline
        \end{tabular}
    \end{footnotesize}
    \vspace{-3mm}
    \label{tab:1}
\end{table}
\egroup