% Parts are the largest structural units, but are optional.
%\part{Thesis}

% Chapters are the next main unit.
\chapter{This is a Chapter}

% Sections are a sub-unit
\section{A Section}
Here is a section with some text.  Equations look like this
$y=x$.\footnote{Here is a footnote.}

This is an example of a second paragraph in a section so you can
see how much it is indented by.

% Subsections follow
\subsection{This is a Subsection}
Here is an example of a citation: \cite{Forbes:2006ba}.  The actual
form of the citation is governed by the bibliographystyle.  These
citations are maintained in a BIBTeX file \texttt{sample.bib}.  You
could type these directly into the file.  For an example of the format
to use look at the file \texttt{ubcsample.bbl} after you compile this
file.\footnote{Here is another footnote.}

This is an example of a second paragraph in a subsection so you can
see how much it is indented by.

\subsubsection{This is a Subsubsection}
Here are some more citations \cite{LL3:1977,Peccei:1989,Turner:1999}.
If you use the \texttt{natbib} package with the \verb+sort&compress+
option, then the following citation will look the same as the first
citation in this section: \cite{Turner:1999,Peccei:1989,LL3:1977}.

This is an example of a second paragraph in a subsubsection so you can
see how much it is indented by.

\paragraph{This is a Paragraph}
Paragraphs and subparagraphs are the smallest units of text.  There is
no subsubsubsection etc.

\subparagraph{This is a Subparagraph}
This is the last level of organisation.  If you need more than this,
you should consider reorganizing your work\dots

\begin{equation}
  \mathrm{f}(x)=\int_{-\infty}^{\int_{-\infty}^x
      e^{-\frac{y^2}{2}}\mathrm{d}{y}}e^{-z^2}\mathrm{d}z
\end{equation}

In order to show you what a separate page would look like (i.e. without
a chapter heading) I must type some more text.  Thus I will babble a
bit and keep babbling for at least one more page\ldots  What you
should notice is that the chapter titles appear substantially lower
than the continuing text. Babble babble
babble babble babble babble babble babble babble babble babble babble
babble babble babble babble babble babble babble babble babble babble
babble babble babble babble babble babble babble babble babble babble
babble babble babble babble babble babble babble babble babble.

Babble babble babble babble babble babble babble babble babble babble
babble babble babble babble babble babble babble babble babble babble
babble babble babble babble babble babble babble babble babble babble
babble babble babble babble babble babble babble babble babble babble
babble babble babble babble babble babble babble babble babble babble
babble babble babble babble babble babble babble babble babble babble
babble babble babble babble babble babble babble babble babble babble
babble babble babble babble babble babble babble babble babble babble
babble babble babble babble babble babble babble babble babble babble
babble babble babble babble babble babble babble babble babble babble
babble babble babble babble babble babble babble babble babble babble
babble babble babble babble babble babble babble babble babble babble
babble babble babble babble.

\begin{table}[t]                 % optional [t, b or h];
  \begin{tabular}{|r||r@{.}l|}
    \hline
    Phoenix & \$960 & 35 \\
    \hline
    Calgary & \$250 & 00 \\
    \hline
  \end{tabular}
  \caption[Here is the caption for this wonderful table\ldots]{
    \label{tab:Table1}
    Here is the caption for this wonderful table. It has not been
    centered and the positioning has been specified to be at the top
    of the page.  Thus it appears above the babble rather than below
    where it is defined in the source file.}
\end{table}

% Force a new page: without this, the quote would appear on the
% previous page.
\newpage

\section{Quote}
Here is a quote:
\begin{quote}
  % It is centered
  \begin{center}
    This is a small poem,\\
    a little poem, a Haiku,\\
    to show you how to.\\
    ---Michael M$^{\rm c}$Neil Forbes.
  \end{center}
\end{quote}

This small poem shows several features:
\begin{itemize}
  \item The use of the \verb|quote| and \verb|center| environments.
  \item The \verb|\newpage| command has been used to force a page
        break.  (Sections do not usually start on a new page.)
  \item The pagestyle has been set to suppress the headers using the
        command \verb|\thispagestyle{plain}|.  Note that using
        \verb|\pagestyle{plain}| would have affected all of the subsequent
        pages.
\end{itemize}
\section{Programs}
Here we give an example of a new float as defined using the
\texttt{float} package.  In the preamble we have used the commands
\begin{verbatim}
\floatstyle{ruled}
\newfloat{Program}{htbp}{lop}[chapter]
\end{verbatim}
This creates a ``Program'' environment that may be used for program
fragments.  A sample \texttt{python} program is shown in
Program~\ref{prog:fib}.  (Note that Python places a fairly restrictive
limit on recursion so trying to call this with a large $n$ before
building up the cache is likely to fail unless you increase the
recursion depth.)
\begin{Program}
  \caption{\label{prog:fib} Python program that computes the $n^{\rm
          th}$ Fibonacci number using memoization.}
  \begin{verbatim}
def fib(n,_cache={}):
    if n < 2:
        return 1
    if n in _cache:
        return _cache[n]
    else:
        result = fib(n-1)+fib(n-2)
        _cache[n] = result
        return result
\end{verbatim}
\end{Program}
Instead of using a \texttt{verbatim} environment for your program
chunks, you might like to \texttt{include} them within an
\texttt{alltt} envrironment by including the \verb|\usepackage{alltt}|
package (see page 187 of the \LaTeX{} book).  Another useful package
is the \verb|\usepackage{listings}| which can pretty-print many
different types of source code.

% Force a new page
\newpage
