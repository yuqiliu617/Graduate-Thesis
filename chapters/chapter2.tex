%% Here we provide a short optional argument to \chapter[]{}.  This
%% optional argument will appear in the table of contents.  For long
%% titles, one should use this to give a single-line entry to the
%% table of contents.
\chapter[Another Chapter\ldots]{Another Chapter with a Very Long
  Chapter-name that will Probably Cause Problems}
\label{cha:apple_ref}

This chapter name is very long and does not display properly in the
running headers or in the table of contents.  To deal with this, we
provide a shorter version of the title as the optional argument to the
\verb|\chapter[]{}| command.

For example, this chapter's title and associated table of contents heading and
running header was created with\\
\verb|\chapter[Another Chapter\ldots]{Another Chapter with a Very Long|\\
\verb|Chapter-name that will Probably Cause Problems}|.

Note that, according to the thesis regulations, the heading included
in the table of contents must be a truncation of the actual heading.

This Chapter was used as a demonstration in the Preface for how to
attribute contribution from collaborators.  If there are any such
contributions, details must be included in the Preface.  If you wish,
you may additionally use a footnote such as this.\footnote{This
  chapter is based on work conducted in UBC's Maple Syrup Laboratory
  by Dr. A. Apple, Professor B. Boat, and C. Cat.}

\section{Another Section}
Another bunch of text to demonstrate what this file does.
You might want a list for example:\footnote{Here is a footnote in a
  different chapter.  Footnotes should come after punctuation.}
\begin{itemize}
  \item An item in a list.
  \item Another item in a list.
\end{itemize}

\section*{An Unnumbered Section That is Not Included in the Table of
  Contents}
\begin{figure}[ht]
  \begin{center}
    %% psfrag: comment the following line if not using the psfrag package
    \psfrag{pie makes me happy!}{$\pi$ makes me happy!}
    %% includegraphics: comment the following if not using the graphicx package
    \includegraphics[width=0.4\textwidth]{fig}
    \caption[Happy Face: figure example.]{\label{fig:happy} This is a figure of
      a happy face with a \texttt{psfrag} replacement.  The original figure
      (drawn in xfig and exported to a .eps file) has the text ``pie makes me
      happy!''.  The \texttt{psfrag} package replaces this with ``$\pi$ makes me
      happy!''.  Note: the Makefile compiles the sample using pdf\LaTeX\ which
      cannot use \texttt{psfrag} directly.  For some options that work with
      pdf\LaTeX, please see this discussion:
      \url{http://tex.stackexchange.com/questions/11839}.  For the caption, we
      have used the optional argument for the caption command so that only a
      short version of this caption occurs in the list of figures.}
  \end{center}
\end{figure}
\afterpage{\clearpage}
Here is an example of a figure environment.
Perhaps I should say that the example of a figure can be seen in
Figure~\ref{fig:happy}.  Figure placement can be tricky with \LaTeX\
because figures and tables are treated as ``floats'': text can flow
around them, but if there is not enough space, they will appear later.
To prevent figures from going too far, the
\verb|\afterpage{\clearpage}| command can be used.  This makes sure
that the figure are typeset at the end of the page (possibly appear on
their own on the following pages) and before any subsequent text.

The \verb|\clearpage| forces a page break so that the figure can be
placed, but without the the \verb|\afterpage{}| command, the page
would be broken too early (at the \verb|\clearpage| statement).  The
\verb|\afterpage{}| command tells \LaTeX{} to issue the command after
the present page has been rendered.

\section{Tables}
We have already included one table:~\ref{tab:Table1}.  Another table
is plopped right here.
\begin{table}[ht]
  \begin{center}
    \begin{tabular}{|l||l|l||l|l|}
      \hline
                 & \multicolumn{2}{l|}{Singular} & \multicolumn{2}{l|}{Plural}                              \\
      \cline{2-5}
                 & English                       & \textbf{Gaeilge}            & English & \textbf{Gaeilge} \\
      \hline\hline
      1st Person & at me                         & \textbf{agam}               & at us   & \textbf{againn}  \\
      2nd Person & at you                        & \textbf{agat}               & at you  & \textbf{agaibh}  \\
      3rd Person & at him                        & \textbf{aige}               & at them & \textbf{acu}     \\
                 & at her                        & \textbf{aici}               &         &                  \\
      \hline
    \end{tabular}
    \caption{
      \label{tab:Table2}
      Another table.}
  \end{center}
\end{table}
Well, actually, as with Figures, tables do not
necessarily appear right ``here'' because tables are also ``floats''.
\LaTeX{} puts them where it can.  Because of this, one should refer to
floats by their labels rather than by their location.  This example is
demonstrated by Table~\ref{tab:Table2}.  This one is pretty close,
however.  (Note: you should generally not put tables or figures in the
middle of a paragraph.  This example is for demonstration purposes
only.)

Another useful package is \verb|\usepackage{longtable}| which provides
the \texttt{longtable} environment.  This is nice because it allows
tables to span multiple pages.  Table~\ref{tab:longtable} has been
formatted this way.
\begin{center}
  \begin{longtable}{|l|l|l|}
    \caption{\label{tab:longtable}Feasible triples for
    highly variable Grid}                                                                                                 \\

    \hline \multicolumn{1}{|c|}{\textbf{Time (s)}} &
    \multicolumn{1}{c|}{\textbf{Triple chosen}}    &
    \multicolumn{1}{c|}{\textbf{Other feasible triples}}                                                                  \\ \hline
    \endfirsthead

    \multicolumn{3}{c}%
    {{\bfseries \tablename\ \thetable{} -- continued from previous page}}                                                 \\
    \hline \multicolumn{1}{|c|}{\textbf{Time (s)}} &
    \multicolumn{1}{c|}{\textbf{Triple chosen}}    &
    \multicolumn{1}{c|}{\textbf{Other feasible triples}}                                                                  \\ \hline
    \endhead

    \hline \multicolumn{3}{|r|}{{Continued on next page}}                                                                 \\ \hline
    \endfoot

    \hline \hline
    \endlastfoot

    0                                              & (1, 11, 13725) & (1, 12, 10980), (1, 13, 8235), (2, 2, 0), (3, 1, 0) \\
    274                                            & (1, 12, 10980) & (1, 13, 8235), (2, 2, 0), (2, 3, 0), (3, 1, 0)      \\
    5490                                           & (1, 12, 13725) & (2, 2, 2745), (2, 3, 0), (3, 1, 0)                  \\
    8235                                           & (1, 12, 16470) & (1, 13, 13725), (2, 2, 2745), (2, 3, 0), (3, 1, 0)  \\
    10980                                          & (1, 12, 16470) & (1, 13, 13725), (2, 2, 2745), (2, 3, 0), (3, 1, 0)  \\
    13725                                          & (1, 12, 16470) & (1, 13, 13725), (2, 2, 2745), (2, 3, 0), (3, 1, 0)  \\
    16470                                          & (1, 13, 16470) & (2, 2, 2745), (2, 3, 0), (3, 1, 0)                  \\
    19215                                          & (1, 12, 16470) & (1, 13, 13725), (2, 2, 2745), (2, 3, 0), (3, 1, 0)  \\
    21960                                          & (1, 12, 16470) & (1, 13, 13725), (2, 2, 2745), (2, 3, 0), (3, 1, 0)  \\
    24705                                          & (1, 12, 16470) & (1, 13, 13725), (2, 2, 2745), (2, 3, 0), (3, 1, 0)  \\
    27450                                          & (1, 12, 16470) & (1, 13, 13725), (2, 2, 2745), (2, 3, 0), (3, 1, 0)  \\
    30195                                          & (2, 2, 2745)   & (2, 3, 0), (3, 1, 0)                                \\
    32940                                          & (1, 13, 16470) & (2, 2, 2745), (2, 3, 0), (3, 1, 0)                  \\
    35685                                          & (1, 13, 13725) & (2, 2, 2745), (2, 3, 0), (3, 1, 0)                  \\
    38430                                          & (1, 13, 10980) & (2, 2, 2745), (2, 3, 0), (3, 1, 0)                  \\
    41175                                          & (1, 12, 13725) & (1, 13, 10980), (2, 2, 2745), (2, 3, 0), (3, 1, 0)  \\
    43920                                          & (1, 13, 10980) & (2, 2, 2745), (2, 3, 0), (3, 1, 0)                  \\
    46665                                          & (2, 2, 2745)   & (2, 3, 0), (3, 1, 0)                                \\
    49410                                          & (2, 2, 2745)   & (2, 3, 0), (3, 1, 0)                                \\
    52155                                          & (1, 12, 16470) & (1, 13, 13725), (2, 2, 2745), (2, 3, 0), (3, 1, 0)  \\
    54900                                          & (1, 13, 13725) & (2, 2, 2745), (2, 3, 0), (3, 1, 0)                  \\
    57645                                          & (1, 13, 13725) & (2, 2, 2745), (2, 3, 0), (3, 1, 0)                  \\
    60390                                          & (1, 12, 13725) & (2, 2, 2745), (2, 3, 0), (3, 1, 0)                  \\
    63135                                          & (1, 13, 16470) & (2, 2, 2745), (2, 3, 0), (3, 1, 0)                  \\
    65880                                          & (1, 13, 16470) & (2, 2, 2745), (2, 3, 0), (3, 1, 0)                  \\
    68625                                          & (2, 2, 2745)   & (2, 3, 0), (3, 1, 0)                                \\
    71370                                          & (1, 13, 13725) & (2, 2, 2745), (2, 3, 0), (3, 1, 0)                  \\
    74115                                          & (1, 12, 13725) & (2, 2, 2745), (2, 3, 0), (3, 1, 0)                  \\
    76860                                          & (1, 13, 13725) & (2, 2, 2745), (2, 3, 0), (3, 1, 0)                  \\
    79605                                          & (1, 13, 13725) & (2, 2, 2745), (2, 3, 0), (3, 1, 0)                  \\
    82350                                          & (1, 12, 13725) & (2, 2, 2745), (2, 3, 0), (3, 1, 0)                  \\
    85095                                          & (1, 12, 13725) & (1, 13, 10980), (2, 2, 2745), (2, 3, 0), (3, 1, 0)  \\
    87840                                          & (1, 13, 16470) & (2, 2, 2745), (2, 3, 0), (3, 1, 0)                  \\
    90585                                          & (1, 13, 16470) & (2, 2, 2745), (2, 3, 0), (3, 1, 0)                  \\
    93330                                          & (1, 13, 13725) & (2, 2, 2745), (2, 3, 0), (3, 1, 0)                  \\
    96075                                          & (1, 13, 16470) & (2, 2, 2745), (2, 3, 0), (3, 1, 0)                  \\
    98820                                          & (1, 13, 16470) & (2, 2, 2745), (2, 3, 0), (3, 1, 0)                  \\
    101565                                         & (1, 13, 13725) & (2, 2, 2745), (2, 3, 0), (3, 1, 0)                  \\
    104310                                         & (1, 13, 16470) & (2, 2, 2745), (2, 3, 0), (3, 1, 0)                  \\
    107055                                         & (1, 13, 13725) & (2, 2, 2745), (2, 3, 0), (3, 1, 0)                  \\
    109800                                         & (1, 13, 13725) & (2, 2, 2745), (2, 3, 0), (3, 1, 0)                  \\
    112545                                         & (1, 12, 16470) & (1, 13, 13725), (2, 2, 2745), (2, 3, 0), (3, 1, 0)  \\
    115290                                         & (1, 13, 16470) & (2, 2, 2745), (2, 3, 0), (3, 1, 0)                  \\
    118035                                         & (1, 13, 13725) & (2, 2, 2745), (2, 3, 0), (3, 1, 0)                  \\
    120780                                         & (1, 13, 16470) & (2, 2, 2745), (2, 3, 0), (3, 1, 0)                  \\
    123525                                         & (1, 13, 13725) & (2, 2, 2745), (2, 3, 0), (3, 1, 0)                  \\
    126270                                         & (1, 12, 16470) & (1, 13, 13725), (2, 2, 2745), (2, 3, 0), (3, 1, 0)  \\
    129015                                         & (2, 2, 2745)   & (2, 3, 0), (3, 1, 0)                                \\
    131760                                         & (2, 2, 2745)   & (2, 3, 0), (3, 1, 0)                                \\
    134505                                         & (1, 13, 16470) & (2, 2, 2745), (2, 3, 0), (3, 1, 0)                  \\
    137250                                         & (1, 13, 13725) & (2, 2, 2745), (2, 3, 0), (3, 1, 0)                  \\
    139995                                         & (2, 2, 2745)   & (2, 3, 0), (3, 1, 0)                                \\
    142740                                         & (2, 2, 2745)   & (2, 3, 0), (3, 1, 0)                                \\
    145485                                         & (1, 12, 16470) & (1, 13, 13725), (2, 2, 2745), (2, 3, 0), (3, 1, 0)  \\
    148230                                         & (2, 2, 2745)   & (2, 3, 0), (3, 1, 0)                                \\
    150975                                         & (1, 13, 16470) & (2, 2, 2745), (2, 3, 0), (3, 1, 0)                  \\
    153720                                         & (1, 12, 13725) & (2, 2, 2745), (2, 3, 0), (3, 1, 0)                  \\
    156465                                         & (1, 13, 13725) & (2, 2, 2745), (2, 3, 0), (3, 1, 0)                  \\
    159210                                         & (1, 13, 13725) & (2, 2, 2745), (2, 3, 0), (3, 1, 0)                  \\
    161955                                         & (1, 13, 16470) & (2, 2, 2745), (2, 3, 0), (3, 1, 0)                  \\
    164700                                         & (1, 13, 13725) & (2, 2, 2745), (2, 3, 0), (3, 1, 0)                  \\
  \end{longtable}
\end{center}

\subsection*{An Unnumbered Subsection}
Note that if you use subsections or further divisions under an
unnumbered section, then you should make them unnumbered as well
otherwise you will end up with zeros in the section numbering.
